
\chapter{Thread}
\label{chap:thread}

\section{Multithread Debugging with libthread\_db}
It is harder and harder to to all the debugging work by debugger alone, especially for the
multithreaed programs.  Some thread library vendors provide an external debugging interface
to debugger, and debugger could utilize this interface to collect the information about all
the threads.

\subsection{Thread Debugging Interface}
On the POSIX system, \texttt{libthread\_db.so} is provided, and the general idea of this
implementation is that, \texttt{libthread\_db.so} provides the interface to access threads,
while debugger provides the interface to access its internal information, such as symbol tables,
method to read or write registers.  In one word, they two (\texttt{libthread\_db.so} and debugger
) export their internal state to each other.  They implement this idea via dynamically link shared
object.

\section{From libthread\_db to debugger}

How \texttt{libthread\_db.so} make use of debugger's functionality.

\subsection{How to Utilize This Interface}
In \texttt{libthread\_db.so}, the interface for debugger to export its internal state is undefined
in \texttt{libthread\_db.so}, and this interface should be implemented by the debugger or other utilities,
which would like to use \texttt{libthread\_db.so}.  On \emph{Linux},

\begin{verbatim}
[qiyao@GreenOnly:]$ ldd -r /lib/libthread_db.so.1
undefined symbol: ps_pglobal_lookup     (/lib/libthread_db.so.1)
undefined symbol: ps_pdwrite    (/lib/libthread_db.so.1)
undefined symbol: ps_lsetfpregs (/lib/libthread_db.so.1)
undefined symbol: ps_getpid     (/lib/libthread_db.so.1)
undefined symbol: ps_lsetregs   (/lib/libthread_db.so.1)
undefined symbol: ps_pdread     (/lib/libthread_db.so.1)
undefined symbol: ps_lgetfpregs (/lib/libthread_db.so.1)
undefined symbol: ps_lgetregs   (/lib/libthread_db.so.1)
        linux-gate.so.1 =>  (0x00f9e000)
        libc.so.6 => /lib/libc.so.6 (0x004b8000)
        /lib/ld-linux.so.2 (0x0049b000)
\end{verbatim}

we could find that there are eight functions undefined in \texttt{libthread\_db.so}, and \emph{GDB} defines these
functions in \texttt{gdb/proc-service.c}.  Otherwise, it would fail to link \texttt{libthead\_db.so} at link-time
(via \emph{-lthread\_db} in command line option to linkers) or run-time(via \emph{dlopen}).

When dealing with \texttt{libthread\_db.so}, there are two important data structures, \emph{ps\_prochandle} and 
\emph{td\_thragent\_t}.

Usually, \emph{ps\_prochandle} is also defined at debugger side, instead of \texttt{libthread\_do.so} \footnote{I don't know why.  Need some investigation here}.



Let us look how \emph{GDB} use \emph{ps\_pglobal\_lookup}, here is a snip for debugging \emph{GDB} with \emph{GDB}.
\begin{verbatim}
Breakpoint 3, ps_pglobal_lookup (ph=0x82c1580, obj=0x114324 "libpthread.so.0", name=0x114431 "nptl_version", sym_addr=0xbf908378) at ../../src/gdb/proc-service.c:180
180       ms = lookup_minimal_symbol (name, NULL, NULL);
\end{verbatim}

We could read the \texttt{proc-service.c},
\begin{verbatim}
169 /* Search for the symbol named NAME within the object named OBJ within
170    the target process PH.  If the symbol is found the address of the
171    symbol is stored in SYM_ADDR.  */
172
173 ps_err_e
174 ps_pglobal_lookup (gdb_ps_prochandle_t ph, const char *obj,
175                    const char *name, paddr_t *sym_addr)
176 {
177   struct minimal_symbol *ms;
178
179   /* FIXME: kettenis/2000-09-03: What should we do with OBJ?  */
180   ms = lookup_minimal_symbol (name, NULL, NULL);
181   if (ms == NULL)
182     return PS_NOSYM;
183
184   *sym_addr = SYMBOL_VALUE_ADDRESS (ms);
185   return PS_OK;
186 }
\end{verbatim}

From this piece of code, we could find that, if we want to write a tool to use the functionality
from \texttt{libthread\_db.so}, we should implement the all the stuff about symbol.

It is relatively easy to implement \emph{lookup\_minimal\_symbol}, of course, ignore the efficiency
here.  In ELF specification, we could know that, there is a section named \texttt{.symtab} or \texttt{.dynsym},
and the type of this section is \texttt{SYMTAB} or \texttt{DYNSYM} respecitvely.  User could check these
information via \texttt{readelf -S}

\begin{verbatim}
[qiyao@GreenOnly:~/SourceCode/NGSTAT]$ readelf -S thread-analysis-tool
There are 37 section headers, starting at offset 0x20ff4:

Section Headers:
  [Nr] Name              Type            Addr     Off    Size   ES Flg Lk Inf Al
......
  [ 4] .dynsym           DYNSYM          08048204 000204 0001c0 10   A  5   1  4
.....
  [35] .symtab           SYMTAB          00000000 0215bc 0007d0 10     36  71  4
\end{verbatim}


These information about symbols is well-structured and fixed in length, and they could be easily parsed.

\subsection{Access Target Memory}

The operating system and system library only provide a simple interface to access the state of the debuggee,
and debugger write should wrap this interface, and enrich it.  When dealing with multithreaded program,
debugger writer should be careful.  Since all the libraries in \emph{Linux} are loaded in a lazy mode,
we must make sure that the thread library has been loaded by \emph{loader, ld-2.4.so} when we want to access
the library.

One lesson I learned is that, I want to  get the address of one symbol in thread library.(\emph{nptl\_version}
in \emph{ld-2.4.so}), and try to fetch the content directly.  After nearly one day debugging, I realize that
the \emph{libpthread} has not been loaded by loader.  That is to say, the memory space of \emph{libpthread} could
only be accessed when loader loads it into the process's space.  If the program reference the symbols in \emph{libpthread}
at first, debugger could access the space of \emph{libpthread}

\subsection{Read and Write Registers}
After implement \emph{ps\_get\_thread\_area} in host, the debugger could do some work at the first several steps, with the help of
\emph{ps\_pglobal\_lookup}, \emph{ps\_pdread}, and \emph{ps\_pdwrite}.  Now the debugger could work until operations to read and
write registers are needed.  Since we are dealing with register read or write operations in multithreaed programming enviroment,
so these operations are a little different with that in common conditoins\footnote{I am not sure what is the difference.}.

In \emph{GDB}, \emph{ps\_lgetregs} is written like this,
\begin{verbatim}
/* Get the general registers of LWP LWPID within the target process PH
   and store them in GREGSET.  */

ps_err_e
ps_lgetregs (gdb_ps_prochandle_t ph, lwpid_t lwpid, prgregset_t gregset)
{
  struct cleanup *old_chain = save_inferior_ptid ();

  inferior_ptid = BUILD_LWP (lwpid, ph->pid);

  target_fetch_registers (-1);
  fill_gregset ((gdb_gregset_t *) gregset, -1);

  do_cleanups (old_chain);
  return PS_OK;
}
\end{verbatim}
In \texttt{target.h}, \emph{target\_fetch\_registers} is a macro like this,
\begin{verbatim}
#define	target_fetch_registers(regno)	\
     (*current_target.to_fetch_registers) (regno)
\end{verbatim}

\emph{GDB} will switch \emph{current\_target} to a \textbf{thread specific target} when it begins
to deal with thread, and \texttt{linux-thread-db.c} defines this \textbf{target}.

Set a breakpoint on \emph{thread\_db\_fetch\_registers}, and have a look how it works.
\begin{verbatim}
(top-gdb) bt
#0  ps_lgetregs (ph=0x82c1740, lwpid=7475, gregset=0xbfd07bb0) at ../../src/gdb/proc-service.c:252
#1  0x00111731 in td_ta_map_lwp2thr () from /lib/libthread_db.so.1
#2  0x0011182d in iterate_thread_list () from /lib/libthread_db.so.1
#3  0x00111a85 in td_ta_thr_iter () from /lib/libthread_db.so.1
#4  0x080961d5 in thread_db_find_new_threads () at ../../src/gdb/linux-thread-db.c:1190
\end{verbatim}

In \emph{GDB}, \emph{inferior\_ptid} is a global variable that identify the current thread that monitored
by \emph{GDB}, and it is hard to know which part or component of \emph{GDB}(only about Linux/i386) modify this global variable\footnote{
I try to set watchpoint on \emph{inferior\_ptid} to have a look who modify it, but failed.}.

It seems that I underestimate the complexity of \emph{inferior\_ptid} in \emph{GDB} source tree.  According
to the list of source files that have \textbf{write operation} on it, I find that it is a global variable
that closely coupled with the current structure of \emph{GDB}.

\section{From debugger to libthread\_db}

How debugger make use of \texttt{libthread\_db.so}'s functionality.
\texttt{libthread\_db.so} provides some interfaces for debugger to call to get some internal
events of thread library.  Usually, debugger uses these APIs via \texttt{dlsym}.

\section{attach thread}
In gdb, it seems that "attach a thread" and "attach process" are different.  My understanding is, attach operation in ptrace is
"attach process", but I do not what is "attach a thread", since I am still not clear about the thread model in glibc.

