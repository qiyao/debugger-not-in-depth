
\chapter{Frame}
\label{chap:fram}
Frame is a very basic concept in computer science, especially in compiler
and operating system.  In short, frame is a piece of memory on stack, which
includes various information on runtime method call information, and stores
some calling states.  In toolchain area, compiler and linker generates code for
frame, while debugger peek code and stack, to understand frames.  Ideally, it
is a quite straight forward, however, unfortunately, in practice, it is a very complicated part,
if not most, in debugger design and implementation, because
\begin{enumerate}
  \item Supporting multiple architectures make it complicated,
  \item Supporting multiple operating systems make it complicated,
  \item Supporting multiple calling conventions or ABI make it complicated,
  \item Handling aggressively-optimized program without debug information make it complicated,
  \item Handling some corner cases make it complicated
\end{enumerate}

\section{Frame Scheme Handler}
Taking factors above into account, given a program, there might be multiple \textit{schemes}
of frame co-exists in a single program, and given a certain frame, its scheme is of these
known schemes, but debugger has to analyze which scheme this frame has.  As a debugger writer,
the first step is that we should handle each of these different frame \textit{schemes}, and
try to apply each handler to a given frame to see which one is applicable.

Usually, it is expensive to try to apply each frame handler to a certain frame, to see whether
this frame handler is \textbf{applicable} to this frame.  It is more efficient
if apply some simple and efficient checks to a certain frame, to see whether this frame handler
is \textbf{not applicable} to this frame.  We can call these simple and efficient checks before
frame handler as \textit{filter}.

%% \includegraphics{frame/figure.1}

\section{Frame Content Handler}
Once \textbf{Frame Scheme Handler} is determined, given a frame and its corresponding frame scheme
handler, the next step is to extract necessary or needed information from it.  In general, these
information are needed to know,
\begin{itemize}
  \item the base address of this frame,
  \item the area for local variables,
  \item the area for arguments passing,
\end{itemize}

Usually, these information can be got from \textit{Frame Content Handler}.  In short, \textit{Frame Content Handler}
is to extract information from a give frame according to some conventions, such as ABI.

\begin{figure}[htb]
  \centering
  \includegraphics[width=5in]{frame/unwind_frame.eps}
  \caption{Recursive frame unwinding}
  \label{fig:frame:unwind}
\end{figure}

\textit{Frame Scheme Handler} and \textit{Frame Content Handler} are interleaved recursively,
from these knowledge on different areas on a frame, debugger can extract necessary or
needed information of previous frame of this current frame.  This process is called \textit{unwind},
shown in Figure \ref{fig:frame:unwind}.  \textit{All Frame Scheme Handlers} and \textit{ABI} are
known to debugger ahead of time of debugging.  \texttt{frame N} is input and \texttt{frame N+1} is
output.

\section{Presentation of frame data}

\section{Implementation in GDB}
The implementation in GDB is quite similar to what we described in previous sections.  Function pointers
\texttt{sniffer} chained in \texttt{frame\_base\_table\_entry} is designed as \textit{Frame Scheme Handler},
while \texttt{frame\_base} returned by \texttt{sniffer} is designed as \textit{Frame Content Handler}.

\begin{figure}[htb]
  \centering
  \includegraphics[width=5.5in]{frame/gdb_frame.eps}
  \caption{GDB frame}
  \label{fig:frame:gdb}
\end{figure}

The workflow of identifying \texttt{frame\_base} is, a frame presentation, \texttt{frame\_info}, is an input,
GDB will go through the linked list of \texttt{frame\_base\_table\_entry}, and call \texttt{sniffer} with
parameter of \texttt{frame\_info}.  If \texttt{sniffer} returns a valid \texttt{frame\_base}, terminate loop,
and use this \texttt{frame\_base} to analyze this frame.