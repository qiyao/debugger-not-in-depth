
\documentclass[12pt]{book}


\usepackage{CJKutf8}
%\usepackage{CJK}
\usepackage{CJKulem}
%\begin{CJK*}{GBK}{song}
\begin{CJK}{UTF8}{gbsn}

\title{Debugger Not In Depth}
\author{Yao Qi} 
\end{CJK}


\usepackage{graphicx}
\usepackage[bookmarks,colorlinks]{hyperref}
\usepackage{color}
\usepackage{verbatim}
\usepackage[bf,small,center,indentafter,pagestyles]{titlesec}
\usepackage{titletoc}
\usepackage{lscape}

\usepackage{listings}
\lstloadlanguages{C}

\usepackage{ifthen}
\newboolean{ChineseLanguage}
\setboolean{ChineseLanguage}{false}

\begin{document}
\setlength{\leftmargin}{60pt}
\setlength{\rightmargin}{0cm}
\setlength{\parsep}{0ex}
\setlength{\itemsep}{0ex}
\setlength{\itemindent}{0pt}
\setlength{\listparindent}{22pt}
\setlength{\topsep}{5pt}
\setlength{\baselineskip}{1.3\baselineskip}

\titleformat{\chapter}[hang]{\centering\LARGE\bfseries}{\chaptername}{1em}{}
\renewcommand{\chaptername}{Chapter {\thechapter}}
\titlespacing{\chapter}{0pt}{*0}{*4}
\titlelabel{\S\thetitle\quad}


\titlecontents{chapter}[0pt]{\vspace{.5\baselineskip}\bfseries}
{Chapter {\thecontentslabel}\quad}{}
{\hspace{.5em}\titlerule*[10pt]{$\cdot$}\contentspage}

\titlecontents{section}[2em]{\vspace{.25\baselineskip}}
{\S\thecontentslabel\quad}{}
{\hspace{.5em}\titlerule*[10pt]{$\cdot$}\contentspage}

%\sethead{}{}{\small\chaptername\quad\chaptertitle\qquad\thepage}

%\begin{CJK*}{GBK}{song}
\begin{CJK}{UTF8}{gbsn}
\maketitle


\lstset{
    basicstyle=\scriptsize,          % print whole listing small
    showstringspaces=false,
%    labelstep=1,
%    commentstyle=\textsl,
    keywordstyle=\color{green}\bfseries\underbar,     % underlined bold red keywords
    identifierstyle={},         % nothing happens to other identifiers
    commentstyle=\mdseries\color{blue}, % white comments
    stringstyle=\color{red}\ttfamily}      % typewriter type for strings
%    stringspaces=false}         % no special string spaces

\frontmatter

\chapter*{Preface}
\label{chap:preface}
%����������Ϊһ��ϵͳ������һ�㲻�ȱ��������߲���ϵͳ�򵥡��ڿ����������Ĺ����У����߸е������ڵ��������ĵ��൱���ޣ������й�����ʦ�������е��ϿɶȲ��ߣ����Ա��߰����Ե������������޵�����д�������ܽ��Ƕ�ʱ����ջ���ϣ�����й���ϵͳ��������ʦ��Щ������

%���ߴ�2003��׿�ʼ�Ӵ�\emph{GNU/Linux}��2005�꿪ʼ��Դ���������ʼϲ����Դ�����ķ�ʽ��ϵͳ������������\emph{Linux}�ں�ת�뵽
%�������ͱ������йص�ϵͳ������ѧϰ���о�������ı�ҵ�����Ŀ���Ǻ͵������йأ�����������һ�ҹ�˾��\emph{Linux}����ʵϰ��������ֱ��
%2006��11�£��Ӵ��˸����ϵͳ������֪ʶ�������ڼ䣬���߲�����һ���µ������Ŀ���������ͬʱ��\emph{GDB}���˷����ͱȽϡ�

I joined \emph{GNU/Linux} at the end of 2003, and started some activities in open source
community on 2005.  Then, I fell
in love with systematic software, so that I switch the area of learning and studying from \emph{Linux} kernel to debugger
and compiler.  I completed my thesis relative to debugger from 2005 to 2006, and have been an intern and a regular later in 
one \emph{Linux} team in one company until Dec. 2006.  During this period, I get much more knowledge about system, and
joined a project about a new debugger \href{http://sourceware.org/frysk/}{frysk} development.  I gradually realized that, debugger, as a systematic software, is
as complicated as operating system or compiler, and there are some reasons for this,

\begin{itemize}
\item Complicated System Runtime features\\
One of the tasks of debugger is to control or monitor the execution of program, and collect the dynamic information of program.
For example, when a program open a shared library dynamically, debugger should be aware this event, and do some operations 
correspondingly(i.e. cache the symbol table of the shared library).
%��������ҪĿ���ǿ��Ƴ�������У�������Ҫ���Ƴ�������״̬���ռ�����״̬�仯��Ϣ�����磬�����������ж�̬����ij�������⣬�����������а취���
%����¼�����������Ӧ�Ĵ��������湲����ķ��ű�����
\item Complicated Multi-platform Support \\
Design and implementation of a debugger is highly coupled with processor, operating system, and system libraries, so a debugger
that support multi-platform must be complicated.
%����������ƺ�ʵ���봦����������ϵͳ��ϵͳ��������أ����ԣ�һ��֧�ֶ��ֲ���ϵͳ�ʹ�������ϵĵ�����һ���ܸ��ӣ�Ҳ�ñ����ܼ򵥵�һ��ʵ�ַ�ʽ
%��ĺܸ��ӡ�
\end{itemize}

%�������������ԣ�ʹ������һ���������Ĺ���ԭ��ʮ�����ѡ����߰ѵ������ֽ�������ɲ��֣�ÿһ���ֶ���һ���򵥵�С����ʵ�֡���ЩС������ܱ��벢����ѧ��
%Ҳ��ͨ�ã���������˵������Ӧ��ԭ����

Debugger is complicated so that it is hard to understand how debugger works.  I decompose debugger into several components, and
write some small program for these components.  These small programs themselves are \emph{not} 100-percent right from a programmer's
perspective and could noly work on Linux/i386 platform, but they are easy to show the mechanism of debugger.
%���е��κ����⣬��ӭָ�������ߵ������� \href{qiyaoltc@gmail.com}{qiyaoltc@gmail.com}

Any comments on literature errors, and systematic software knowledge are welcome.  Contact me via \href{<Yao Qi>}{qiyaoltc@gmail.com}


\tableofcontents

\mainmatter


\chapter{Introduction}
\label{chap:introduction}
The introduction of the history and category of debugger is out of the 
scope of the book, while this chapter is only focus on,
\begin{itemize}
\item What components are involved?
\item What is a simplest debugger?
\end{itemize}

\section{Components involved}

\begin{enumerate}
\item Special interface, that access the state the program. \\
  In current implementations of UNIX/Linux, cross-process access is highly 
restricted, from the security's perspective, but OS still provides
  a special interface for the father process to access its child process. 
It is called \textbf{ptrace}.  Debugger could not only modify the instructions of program,
  but also check the signals of the program.
\item Operating System, that notifies debugger. \\
  Debugger only does some operations, and then wait these events.  Operating 
system will notify the debugger when some events happened, 
  since in UNIX/Linux, father process has the right to know what happens to 
its child process(just like our real life).
\end{enumerate}

\section{A simple debugger}
In a extremely easy condition, debugging is an inter-process communication 
between father and child process, while father is debugger and child is debuggee.
Debugger will use all the functionality provided by operating system, system 
libraries and others, to control the execution and the state of debuggee.

Here is a simple debugger, too simple to name it debugger,

\lstinputlisting[language=C,caption={hellodebugger.c},captionpos={b},label={listing:introduction:hellodebugger},frame=trbl]{introduction/hellodebugger.c}
\lstinputlisting[language=C,caption={simple-debugger.c},captionpos={b},label={listing:introduction:simple-debugger},frame=trbl]{introduction/simple-debugger.c}

Listing \ref{listing:introduction:simple-debugger} could be separated to three parts, and I think, the general debugger could also be separated in the same way,
\begin{enumerate}
\item Fork a child\\
In order to get the fully control over debuggee, debugger should first 
\textbf{fork} a new child process, and set up the execution image by \textbf{execl} system call.  
\textbf{PTRACE\_TRACEME} tells Linux kernel that this new created child process 
is under \textbf{ptraced}, so the behavior of this process is a little 
different\footnote{Difference are twofold, one is execution synchronization 
with father and one is ptrace interface initialization} with other common process.

\item Wait \\
Debugger calls system call \textbf{wait}, waiting for the events from debuggee
\footnote{Actually, these events are send by operating system}, such as 
hit breakpoint, get a signal and etc., sent by operating system.  Most of the time, 
debugger will be blocked at \textbf{wait()}, until some events happnes.

\item Actions against the current event
Debugger is noified by operating system by some events, and go on its execution 
from \textbf{wait}, and could take some actions according to the type and information 
of the current event.
Actions taken here and internal logic here varies from different debuggers' implementations.  
In this simple case, debugger reads the current program counter register value 
via system call \textbf{ptrace}.
\end{enumerate}


\lstinputlisting[caption={output of simple-debugger},captionpos={b},label={listing:introduction:simple-debugger-output},frame=trbl]{introduction/simple-debugger-output}

The result in Listing \ref{listing:introduction:simple-debugger-output} shows that event 
\texttt{SIGTRAP}\footnote{run \texttt{kill -l} to check the number of different signal} was sent, 
and the value of the program counter register when this event happened.

There is one interesting question that everyone, at least me, may ask when he/she gets this result, 
\texttt{why program counter is this when \textbf{SIGTRAP} happens?}
My answer to this questions is, why not check what this address is?
and then I disassembly \textbf{hellodebugger} line by line.  Unfortunately, 
I could not find this address in its code.  Why?  To be honest, I am confused
by this for some days, until I read the \textbf{ELF} specification\cite{Manual:ELF}.
My understanding is that debugger should know when the debuggee start to
execution its code, otherwise, debugger will miss this control right to debuggee, 
so what is the starting point of debuggee execution?  Is it at \textbf{main()}?  Absolutely not.
System loader and C runtime library will do a lot of thing before \textbf{main()}.
That is to say, debugger starts its monitor even when the execution is still
in system library, or operating system provide this to debugger.
Disassembly \textbf{/lib/ld-2.4.so}, and check this address,

\lstinputlisting[caption={},captionpos={b},label={listing:introduction:dis-ld},frame=trbl]{introduction/dis-ld}

This output tells us that Linux kernel will produce an event when the 
execution of program process reach the entry of child program image,\texttt{\_start}, 
if \textbf{PTRACE\_TRACEME} is called. 
From now on, this process is under the control and monitor of debugger, 
until the death of this process, or debugger give up its monitor on this process.





%%%%%%%%%%%%%%%%%%%%%%%%%%%%%%%%%%%%%%%%%%%%  New chapter  %%%%%%%%%%%%%%%%%%%%%%%%%%%%%%%%%%%%%%%%%%%%%
\chapter{Breakpoint}
\label{chap:breakpoint}
Typically, breakpoints are implemented as a special instruction that cause
a \textbf{trap} to operating system, which then can notifies debugger.  The debugger,
through a special interface routines provided by operating system, has the
ability to read and write the text(executable code) space of the debuggee
process.  Given an address in this text space where a breakpoint needs to be
set , the debugger can read the current instruction at that location and
save it for later replacement.  Then, the debugger writes the special
breakpoint instruction at that location.


\section{What breakpoint is?}
Debuggers only set the breakpoint at the place that programmers are interested in,
and wait for the event corresponding to this breakpoint.  There are two most 
confusing or interesting things,
 
\begin{itemize}
\item What is \textbf{breakpoint} actually?\\
\textbf{breakpoint} is an special or invalid instruction, closely relative the \textbf{instruction set} of the machine.
it is \emph{int3} on \textit{x86} while invalid instruction on \textit{PowerPC}, which will cause the execution of the program \textbf{trap} to operating system.
Debugger writes the special instruction to the program's text space via \textbf{ptrace} to replace the original instruction.

\item How debugger is notified?\\
When program hits this special instruction, written by debugger some times before, 
the execution of program \textbf{trap} to operating system,
and then, operating system would notify the father of this program by means of sending a 
\texttt{SIGTRAP} signal to father process.  Debugger could catch this event via \textbf{wait()}, 
and know that program hits a breakpoint.
\end{itemize}

\section{A hard-coded example}
Here is a simple example to explain what the \textbf{breakpoint} is.
Listing \ref{listing:breakpoint:int3} hardwires special instruction \emph{int3} 
in the form of inline assembly directly.  Of course, there is no program written 
in this way, but this example could demonstrate how debugger deal with event caused
by breakpoint hit, without dealing with details on how to \textbf{ptrace} operations
and symbol resolution.

%\renewcommand\baselinestretch{0.5}\selectfont
\lstinputlisting[language=C,caption={int3.c},captionpos={b},label={listing:breakpoint:int3},frame=trbl]{breakpoint/int3.c}
%\par
\lstinputlisting[language=C,caption={simple-break.c},captionpos={b},label={listing:breakpoint:simple-break},frame=trbl]{breakpoint/simple-break.c}
\lstinputlisting[frame=trbl,caption={output of simple-break},captionpos={b},label={listing:breakpoint:simple-break-output}]{breakpoint/simple-break-output}

Listing \ref{listing:breakpoint:simple-break-output} shows that event \texttt{SIGTRAP} was sent twice, 
and the value of the program counter register when these two event happened.  
We could disassembly program \textbf{int3} to have a look at what these two address are.  The first program counter register value has been discussed in chapter 
\ref{chap:introduction} for Listing \ref{listing:introduction:simple-debugger-output}, 
and the second program counter register value is at the address
\footnote{Execution of assembly \textbf{int3} causes this event, and program counter register contains the NEXT instruction that will be executed.
The value of program counter register is bigger than the address of \textbf{int3} by 1.} of assembly \textbf{int3}.

\lstinputlisting[frame=trbl,caption={Disassembly the main() function},captionpos={b},label={listing:breakpoint:dis-int3}]{breakpoint/dis-int3}

From the debugger's perspective, \textbf{breakpoint mechanism} is simply that 
debugger modifies the program's text space to insert some special instruction, 
and wait for the notification with the help of operating system and hardware processor.
However, debugger should not change the behaviour of the program execution, 
so debugger always save the original bytes
where it plans to set breakpoint, and restore it when necessary.  

When breakpoint hits, the programmers will check the information they are interested in, 
such as register values, variables, stacks and so forth,
and then they would like to continue the execution of the program as if 
the program execution is not interrupted by this breakpoint.  Debugger will do several steps
as follows to resume the execution of program,

\begin{enumerate}
\item Restore the byte(s) of original instruction
\item Decrease program counter register by the length of this special instruction
\item Single step one instruction
\item Write the special instruction back to avoid breakpoint lost, 
since programer does not intend to remove this breakpoint.
\item Resume the program in a full speed
\end{enumerate}

\begin{figure}[htb]
  \centering
  \includegraphics{breakpoint/break.1}\hfill\includegraphics{breakpoint/break.2}\hfill\includegraphics{breakpoint/break.3}\hfill
  \includegraphics{breakpoint/break.4}\hfill\includegraphics{breakpoint/break.5}\hfill\includegraphics{breakpoint/break.6}
  \caption{Set breakpoint and resume execution}
  \label{fig:break}
\end{figure}

\section{Breakpoint operations}
It is hard to imagine that applications developers spin the breakpoint in their source code 
when they write code, so debugger should has the capability to insert or remove breakpoint on-the-fly.
From the end users' perspective, they mostly insert breakpoints on function, source code line, 
or address, and remove them.  Debugger could map function and source line to address, by means of symbol
resolve, which will be covered in Chapter \ref{chap:symbol}, and then create the bi-relationship between breakpoints and address.
Debugger will maintain a list of breakpoints of different types, and provide a simple interface of 
breakpoint operations for other components of debugger\footnote{for example, shared object sub-system is dependent on breakpoint mechanism to detect the event of loading and unloading a shared library.}

As mentioned above, breakpoint operations are indeed the access to child's memory space and registers via \textbf{ptrace} system call, so breakpoint operations are actually a series of \textbf{ptrace} system calls, with different options.

Source code in Listing \ref{listing:breakpoint:break-ops} shows three operations about breakpoint, \texttt{setup\_breakpoint}, \texttt{restore\_breakpoint} and
\texttt{hit\_breakpoint}.  In \texttt{main()} function, father process sets a breakpoint on \texttt{0x004a93c9}, which is the address of symbol 
\texttt{\_dl\_debug\_state}\footnote{The address of this symbol varies from distros and hardware platform.  Reader could check the address of this symbol on your
distro when you want to run it program.  I hardwire this address in order to avoid too much details on symbol and ELF, which are covered in Chapter \ref{chap:symbol} 
and Chapter \ref{chap:elf}} in \texttt{ld-2.4.so}.

\lstinputlisting[language=C,caption={break-ops.c},captionpos={b},label={listing:breakpoint:break-ops},frame=trbl]{breakpoint/break-ops.c}

From the source code, we could find that breakpoint operations are highly relative to \textbf{ptrace} operations, and it is not difficult as you imagine when you are familiar with \textbf{ptrace}.  From the output, we could find that breakpoint on \texttt{\_dl\_debug\_state} is hit for sometimes.
\lstinputlisting[caption={output of break-ops},captionpos={b},label={listing:breakpoint:break-ops-output},frame=trbl]{breakpoint/break-ops-output}

%%%%%%%%%%%%%%%%%%%%%%%%%%%%%%%%%%%%%%%%%%%%%%%%%%%%%%%%%%%  ELF  %%%%%%%%%%%%%%%%%%%%%%%%%%%%%%%%%%%%%%%%%%%%%%%%%%%%%%%%%%%%%%%%%%%%%%%%%%%%%%%%%%%
%%%%%%%%%%%%%%%%%%%%%%%%%%%%%%%%%%%%%%%%%%%%%%%%%%%%%%%%%%%%%%%%%%%%%%%%%%%%%%%%%%%%%%%%%%%%%%%%%%%%%%%%%%%%%%%%%%%%%%%%%%%%%%%%%%%%%%%%%%%%%%%%%%%%%
\chapter{ELF}
\label{chap:elf}

\include{dwarf}

\section{阅读理解DWARF标准}
\section{GDB中对dwarf的处理}
%%%%%%%%%%%%%%%%%%%%%%%%%%%%%%%%%%%%%%%%%%%%%%%%%%%%%%%%%%%  Frame  %%%%%%%%%%%%%%%%%%%%%%%%%%%%%%%%%%%%%%%%%%%%%%%%%%%%%%%%%%%%%%%%%%%%%%%%%%%%%%%%%
%%%%%%%%%%%%%%%%%%%%%%%%%%%%%%%%%%%%%%%%%%%%%%%%%%%%%%%%%%%%%%%%%%%%%%%%%%%%%%%%%%%%%%%%%%%%%%%%%%%%%%%%%%%%%%%%%%%%%%%%%%%%%%%%%%%%%%%%%%%%%%%%%%%%%
%\chapter{Frame}
%\label{chap:frame}

\chapter{Frame}
\label{chap:fram}
Frame is a very basic concept in computer science, especially in compiler
and operating system.  In short, frame is a piece of memory on stack, which
includes various information on runtime method call information, and stores
some calling states.  In toolchain area, compiler and linker generates code for
frame, while debugger peek code and stack, to understand frames.  Ideally, it
is a quite straight forward, however, unfortunately, in practice, it is a very complicated part,
if not most, in debugger design and implementation, because
\begin{enumerate}
  \item Supporting multiple architectures make it complicated,
  \item Supporting multiple operating systems make it complicated,
  \item Supporting multiple calling conventions or ABI make it complicated,
  \item Handling aggressively-optimized program without debug information make it complicated,
  \item Handling some corner cases make it complicated
\end{enumerate}

\section{Frame Scheme Handler}
Taking factors above into account, given a program, there might be multiple \textit{schemes}
of frame co-exists in a single program, and given a certain frame, its scheme is of these
known schemes, but debugger has to analyze which scheme this frame has.  As a debugger writer,
the first step is that we should handle each of these different frame \textit{schemes}, and
try to apply each handler to a given frame to see which one is applicable.

Usually, it is expensive to try to apply each frame handler to a certain frame, to see whether
this frame handler is \textbf{applicable} to this frame.  It is more efficient
if apply some simple and efficient checks to a certain frame, to see whether this frame handler
is \textbf{not applicable} to this frame.  We can call these simple and efficient checks before
frame handler as \textit{filter}.

%% \includegraphics{frame/figure.1}

\section{Frame Content Handler}
Once \textbf{Frame Scheme Handler} is determined, given a frame and its corresponding frame scheme
handler, the next step is to extract necessary or needed information from it.  In general, these
information are needed to know,
\begin{itemize}
  \item the base address of this frame,
  \item the area for local variables,
  \item the area for arguments passing,
\end{itemize}

Usually, these information can be got from \textit{Frame Content Handler}.  In short, \textit{Frame Content Handler}
is to extract information from a give frame according to some conventions, such as ABI.

\begin{figure}[htb]
  \centering
  \includegraphics[width=5in]{frame/unwind_frame.eps}
  \caption{Recursive frame unwinding}
  \label{fig:frame:unwind}
\end{figure}

\textit{Frame Scheme Handler} and \textit{Frame Content Handler} are interleaved recursively,
from these knowledge on different areas on a frame, debugger can extract necessary or
needed information of previous frame of this current frame.  This process is called \textit{unwind},
shown in Figure \ref{fig:frame:unwind}.  \textit{All Frame Scheme Handlers} and \textit{ABI} are
known to debugger ahead of time of debugging.  \texttt{frame N} is input and \texttt{frame N+1} is
output.

\section{Presentation of frame data}

\section{Implementation in GDB}
The implementation in GDB is quite similar to what we described in previous sections.  Function pointers
\texttt{sniffer} chained in \texttt{frame\_base\_table\_entry} is designed as \textit{Frame Scheme Handler},
while \texttt{frame\_base} returned by \texttt{sniffer} is designed as \textit{Frame Content Handler}.

\begin{figure}[htb]
  \centering
  \includegraphics[width=5.5in]{frame/gdb_frame.eps}
  \caption{GDB frame}
  \label{fig:frame:gdb}
\end{figure}

The workflow of identifying \texttt{frame\_base} is, a frame presentation, \texttt{frame\_info}, is an input,
GDB will go through the linked list of \texttt{frame\_base\_table\_entry}, and call \texttt{sniffer} with
parameter of \texttt{frame\_info}.  If \texttt{sniffer} returns a valid \texttt{frame\_base}, terminate loop,
and use this \texttt{frame\_base} to analyze this frame.

 
\chapter{Symbol}
\label{chap:symbol}
Application developers do not know which line in a source code they want to set breakpoints when
they write the source, and debugger could set or remove breakpoints on-the-fly.

%%%%%%%%%%%%%%%%%%%%%%%%%%%%%%%%%%%%%%%%%%%%%%%%%%%%%%%%%%%%%%%%%%%%%%%%%%%%%%%%%%%%%%%%%%%%%%%%%%%%%%%%%%%%%%%%%%%%%%%%%%%%%%%%%%%%%%%
%%%%%%%%%%%%%%%%%%%%%%%%%%%%%%%%%%%%%%%%%%%%%%%%%%%%%  Shared Object  %%%%%%%%%%%%%%%%%%%%%%%%%%%%%%%%%%%%%%%%%%%%%%%%%%%%%%%%%%%%%%%%%%

\chapter{Shared Object}
\label{chap:sharedobject}
From the chapter \ref{chap:introduction}, we know that debugger should monitor all the 
actions or events that may change the state of program.
Loading and unloading a shared object at run-time is one of these events.
Shared object also obeys the ELF specification \cite{Manual:ELF}(Actually, shared object
is one of three format in ELF specification), and the
static feature and file format could discussed in Chapter \ref{chap:elf}.  
In this chapter, we only think about how does debugger know when a
shared object is loaded or unloaded at runtime.

It is one of the most attractive features of UNIX SVR4 that shared object could be loaded and unloaded at run-time dynamically.  This feature reduces the size
of code, and increases the independence between applications and libraries, but it makes troubles for the debuggers.  Debuggers could not only
cover all the data and code of the debuggee, but also should pay attention on the new code and data in shared obect may be loaded during the execution of this program.
Debugger should know,
\begin{itemize}
\item When is a shared object loaded? or how a debugger is notified when a shared object is loaded?
\item How a debugger get the information about shared object loaded in the current process? What is loaded?
\end{itemize}

The problem is system runtime libraries and dynamic linker, which are responsible for loading and unloading shared objects at runtime, do not know how to notify debuggers,
in other words, system libraries and dynamic linker do not know the existence of debuggers\footnote{One of the UNIX philosophy is that do one thing and do one thing well.  
Some additional work are needed for the co-work between different components.}.  They could not help debuggers on this issue, and debuggers
have to do something by themselves to \texttt{force} system runtime libraries and dynamic linker to notify these event that debuggers want.  
Since all the shared object are loaded and unloaded by dynamic linker, \texttt{/lib/ld-2.4.so} on \textbf{Linux}, debugger
could know these operatoins and information if it monitors some key points or probe points in dynamic linker.
In the current implementations of \textbf{GDB}, some breakpoints are set on some key points in dynamic linker.
When dynamic linker begins to load a library, it will hit a breakpoint \texttt{pre-set} by \textbf{GDB}, 
and \textbf{GDB} could receive this event, and it could know when a library is being loaded and unloaded, and what is being loaded and unloaded.

At the same time, dynamic linker exposes some public interfaces or \textbf{probe points} for debuggers to set breakpoints on, in order to increase the debuggability 
of dynamic linker when deal with shared object actions or events.  If dynamic linker does not expose an interface explicitly, debugger writers have to set breakpoint 
according to their own
understanding, and break the consistence of system library and dynamic linker.  In the current implementation of \textbf{GNU C Library}, it exposes
a symbol \texttt{\_dl\_debug\_state} explicitly for debugger to set breakpoint on, and make sure that the breakpoint on this function is hit, 
when the system runtime library is at a consistent state.  Due to the efforts made at the two sides, system library expose a symbol for debugger, and make sure that
it is in a consistent state when debuggee's execution reaches to this point, while debugger set breakpoint on the address of this symbol, 
and collect information when execution hits this breakpoint.

Setting a breakpoint at \texttt{\_dl\_debug\_state} in dynamic linker could only enable the debuggers could notified when a library is loaded and unloaded,
but debuggers should make additional negotiation with dynamic linker and system runtime library that system library should maintain a list of libraries that
has been loaded and export interface for debuggers to access this list, if debuggers want to know what libraries are being loaded.

Listing \ref{listing:sharedobject:test-dl} is a simple program that call \texttt{dlopen} to open a shared library dynamically, and our sample program
in Listing \ref{listing:sharedobject:break-loader} could detect all the events about library load and unload.

\lstinputlisting[language=C,caption={test-dl.c},captionpos={b},label={listing:sharedobject:test-dl},frame=trbl]{sharedlibrary/test-dl.c}
\lstinputlisting[language=C,caption={link-map.c},captionpos={b},label={listing:sharedobject:link-map},frame=trbl]{sharedlibrary/link-map.h}
\lstinputlisting[language=C,caption={break-loader.c},captionpos={b},label={listing:sharedobject:break-loader},frame=trbl]{sharedlibrary/break-loader.c}

Listing \ref{listing:sharedobject:break-loader-output} shows the output when we run \texttt{break-loader},
\begin{landscape}
\lstinputlisting[caption={output of break-loader},captionpos={b},label={listing:sharedobject:break-loader-output},frame=trbl]{sharedlibrary/break-loader-output}
\end{landscape}

Although Listing \ref{listing:sharedobject:break-loader-output} looks a little in mess, at least my feeling is, when I got it for the first time, we still get 
some interesting information from the debugger's perspective or system developer's perspective,
\begin{itemize}
\item Relocation information.
There are three elements in every records, \texttt{l\_addr}, \texttt{l\_ld}, and \texttt{l\_name}.  We could get the meaning of \texttt{l\_addr} and \texttt{l\_ld}
from the comments to them in Listing \ref{listing:sharedobject:link-map}.  \texttt{l\_addr}, which is set by runtime library, could tell debugger where this
library is loaded at, and debugger could adjust the address of every symbol that is relocated.
\item Optimizations to DSO.
It is confusing that \texttt{l\_addr} is $0$ in some records, and that is what this item addresses.  Since \textsf{GNU C Library 2.3}, some optimizations are
introduced into it, and \textbf{pre-linking}\cite{Manual:glibc2.3} is one of them.  If \textbf{pre-linking} optimization is applied to this library or shared
object, and \textbf{prelink} could find a suitable load address for this library, \texttt{l\_addr} will be $0$, and this library is loaded at that fixed address
set by \textbf{prelink}, in order to speed up library loading. 
\end{itemize}

We have discussed about how debugger check the shared object loaded in this process's space.  Actually, after checking the list of link map in \textbf{glibc},
there are still a lot of things to be done by debugger, such as relocation and symbol resolve, but these topics are out of the scope of this chapter, and
will be discussed in chapter \ref{chap:symbol} partially. 

\subsection{Shared library and Breakpoint}
Shared library is nearly identical to other \emph{ELF} object files, except its special features on loading, symbol
resolution, and others.(My point is other special features are affected by its loading feature, so we could pay attention
on the root cause, \emph{dynamic load}).  From a debugger's perspective, the difference between shared library and other \emph{ELF}
object file is debugger do not know when the shared library is loaded by \emph{dynamic linker}, so debugger should monitor
every operation of \emph{dynamic linker}, so that record and maintain the information of shared library loaded in debuggee's space.
\emph{GDB} monitors every operation of \emph{dynamic linker} by means of setting \emph{breakpoints} in some functions in \emph{dynamic linker}.


When \emph{GDB} starts up, it will load \emph{/lib/ld-linux.so.2}\footnote{I do not know why gdb does this}.  I set a breakpoint on 
\emph{solib\_open} in \texttt{gdb/solib.c}, and get the result like this,

\begin{verbatim}
Breakpoint 4, solib_open (in_pathname=0xbfebda00 "/lib/ld-linux.so.2", found_pathname=0xbfebda58) at ../../src/gdb/solib.c:145
145       struct target_so_ops *ops = solib_ops (current_gdbarch);
(top-gdb) bt
#0  solib_open (in_pathname=0xbfebda00 "/lib/ld-linux.so.2", found_pathname=0xbfebda58) at ../../src/gdb/solib.c:145
#1  0x080915b6 in enable_break () at ../../src/gdb/solib-svr4.c:982
\end{verbatim}

Here is a important data structure relative to shared library,(imo, it is the interface to solib)
\begin{verbatim}
struct target_so_ops {
    void (*relocate_section_addresses)(struct so_list *, struct section_table *);
    void (*free_so)(struct so_list *);
    void (*clear_solib)(void);
    void (*solib_create_inferior_hook)(void);
    void (*special_symbol_handling)(void);
    struct so_list *(*current_sos)(void);
    int (*open_symbol_file_object)(void *);
    int (*in_dynsym_resolve_code)(CORE_ADDR);
    int (*find_and_open_solib)(char *, unsigned int, char **);
} *
(top-gdb) p ops[0]
$5 = {
relocate_section_addresses = 0x80914c0 <svr4_relocate_section_addresses>, 
free_so = 0x8090990 <svr4_free_so>, 
clear_solib = 0x8090730 <svr4_clear_solib>, 
solib_create_inferior_hook = 0x8091960 <svr4_solib_create_inferior_hook>, 
special_symbol_handling = 0x8090720 <svr4_special_symbol_handling>, 
current_sos = 0x8090ff0 <svr4_current_sos>,
open_symbol_file_object = 0x8090cf0 <open_symbol_file_object>, 
in_dynsym_resolve_code = 0x8090940 <svr4_in_dynsym_resolve_code>, 
find_and_open_solib = 0}

\end{verbatim}

In \texttt{gdb/solib-svr4.c}\footnote{\emph{svr4} stands for \emph{System V Release 4}.  It was the most successful version of \emph{UNIX}.}, 
we could find a list of symbols in the dynamic linker.
\begin{verbatim}
static char *solib_break_names[] =
{
  "r_debug_state",
  "_r_debug_state",
  "_dl_debug_state",
  "rtld_db_dlactivity",
  "_rtld_debug_state",

  /* On the 64-bit PowerPC, the linker symbol with the same name as
     the C function points to a function descriptor, not to the entry
     point.  The linker symbol whose name is the C function name
     prefixed with a '.' points to the function's entry point.  So
     when we look through this table, we ignore symbols that point
     into the data section (thus skipping the descriptor's symbol),
     and eventually try this one, giving us the real entry point
     address.  */
  "._dl_debug_state",

  NULL
};
\end{verbatim}

On \emph{Linux}, we could find this symbol \emph{\_dl\_debug\_state}
in \emph{/lib/ld-2.4.so},
\begin{verbatim}
[qiyao@GreenOnly:~]$ readelf -s /lib/ld-2.4.so  | grep _dl_debug_state
    22: 004a93c9     5 FUNC    GLOBAL DEFAULT    9 _dl_debug_state@@GLIBC_PRIVATE
   410: 004a93c9     5 FUNC    LOCAL  HIDDEN    9 __GI__dl_debug_state
   451: 004a93c9     5 FUNC    GLOBAL DEFAULT    9 _dl_debug_state
\end{verbatim}

\emph{\_dl\_debug\_state} is defined in \emph{glibc/elf/dl-debug.c}, and the comments
to this function are useful to understand it.  \texttt{This function exists solely to have a 
breakpoint set on it by the debugger.  The debugger is supposed to find this function's 
address by examining the \emph{r\_brk} member of \emph{struct r\_debug}, but GDB 4.15 in fact looks
for this particular symbol name in the \emph{PT\_INTERP} file.}


Now, we could know that debugger set breakpoint on \emph{\_dl\_debug\_state} to monitor the operations
of \emph{dynamic linker}, and in the following part, we will investigation what does debugger do, when
the program hit this breakpoint.


This part of code may could tell us how debugger deal with shared library(\emph{libpthread.so.o} here)
\begin{verbatim}
Breakpoint 4, solib_open (in_pathname=0xa135548 "/lib/libpthread.so.0", found_pathname=0xbfebd7a8) at ../../src/gdb/solib.c:145
145       struct target_so_ops *ops = solib_ops (current_gdbarch);
(top-gdb) bt
#0  solib_open (in_pathname=0xa135548 "/lib/libpthread.so.0", found_pathname=0xbfebd7a8) at ../../src/gdb/solib.c:145
#1  0x08090540 in solib_map_sections (arg=0xa1a3828) at ../../src/gdb/solib.c:272
#2  0x081179d3 in catch_errors (func=0x8090500 <solib_map_sections>, func_args=0xa1a3828, errstring=0x823d478 "Error while mapping shared library sections:\n", mask=6)
    at ../../src/gdb/exceptions.c:515
#3  0x0808fdfc in update_solib_list (from_tty=0, target=0x82da3c0) at ../../src/gdb/solib.c:582
#4  0x080900d8 in solib_add (pattern=0x0, from_tty=0, target=0x82da3c0, readsyms=1) at ../../src/gdb/solib.c:643
#5  0x0810e235 in handle_inferior_event (ecs=0xbfebd9cc) at ../../src/gdb/infrun.c:2178
\end{verbatim}

\begin{verbatim}
Breakpoint 5, svr4_fetch_link_map_offsets () at ../../src/gdb/solib-svr4.c:1421
1421      struct solib_svr4_ops *ops = gdbarch_data (current_gdbarch, solib_svr4_data);
(top-gdb) bt
#0  svr4_fetch_link_map_offsets () at ../../src/gdb/solib-svr4.c:1421
During symbol reading, Incomplete CFI data; unspecified register eax at 0x08090a23.
#1  0x08090ccb in solib_svr4_r_map () at ../../src/gdb/solib-svr4.c:555
#2  0x08091012 in svr4_current_sos () at ../../src/gdb/solib-svr4.c:697
#3  0x0808fd2e in update_solib_list (from_tty=0, target=0x82da3c0) at ../../src/gdb/solib.c:470
#4  0x080900d8 in solib_add (pattern=0x0, from_tty=0, target=0x82da3c0, readsyms=1) at ../../src/gdb/solib.c:643
#5  0x0810e235 in handle_inferior_event (ecs=0xbfa55d6c) at ../../src/gdb/infrun.c:2178
\end{verbatim}
That is to say, \emph{svr4\_fetch\_link\_map\_offsets} actually call \emph{svr4\_ilp32\_fetch\_link\_map\_offsets}


In \emph{solib\_svr4\_r\_map()}, we could go on to analyze this code,
\begin{verbatim}
static CORE_ADDR
solib_svr4_r_map (void)
  {
     struct link_map_offsets *lmo = svr4_fetch_link_map_offsets ();
     return read_memory_typed_address (debug_base + lmo->r_map_offset,
                                         builtin_type_void_data_ptr);
  }
\end{verbatim}
\emph{debug\_base} is \emph{\_r\_debug} in \emph{ld-2.4.so}
\begin{verbatim}
(top-gdb) p/x debug_base
$2 = 0x4b56a4
\end{verbatim}
\begin{verbatim}
[qiyao@GreenOnly:]$ readelf -s /lib/ld-2.4.so | grep 4b56a4
    35: 004b56a4    20 OBJECT  GLOBAL DEFAULT   19 _r_debug@@GLIBC_2.0
   464: 004b56a4    20 OBJECT  GLOBAL DEFAULT   19 _r_debug
\end{verbatim}

And the return value of \emph{solib\_svr4\_r\_map} is the address of the first entry,
\begin{verbatim}
(top-gdb) p/x lm
$3 = 0x4b56b8
\end{verbatim}
\begin{verbatim}
[qiyao@GreenOnly:]$ readelf -s /lib/ld-2.4.so | grep 4b56b8
   406: 004b56b8     0 NOTYPE  LOCAL  DEFAULT  ABS _end
\end{verbatim}

\begin{figure}[htb]
\centering

\includegraphics[width=12cm]{figure.1}
\caption{Data Structures about Shared Object Debuggin in GDB}
\label{sods}
\end{figure}

\emph{ps\_get\_thread\_area} should be defined by debugger, otherwise, an error, \emph{NO CAPABILITY}, will return.
In \emph{glibc/nptl\_db/td\_ta\_map\_lwp2thr.c}, we could see that,
\begin{verbatim}
142 # pragma weak ps_get_thread_area

153      case ta_howto_reg_thread_area:
154       if (&ps_get_thread_area == NULL)
155         return TD_NOCAPAB;
\end{verbatim}
If \emph{ps\_get\_thread\_area} is not defined in host, return an error.


\chapter{Thread}
\label{chap:thread}

\section{Multithread Debugging with libthread\_db}
It is harder and harder to to all the debugging work by debugger alone, especially for the
multithreaed programs.  Some thread library vendors provide an external debugging interface
to debugger, and debugger could utilize this interface to collect the information about all
the threads.

\subsection{Thread Debugging Interface}
On the POSIX system, \texttt{libthread\_db.so} is provided, and the general idea of this
implementation is that, \texttt{libthread\_db.so} provides the interface to access threads,
while debugger provides the interface to access its internal information, such as symbol tables,
method to read or write registers.  In one word, they two (\texttt{libthread\_db.so} and debugger
) export their internal state to each other.  They implement this idea via dynamically link shared
object.

\section{From libthread\_db to debugger}

How \texttt{libthread\_db.so} make use of debugger's functionality.

\subsection{How to Utilize This Interface}
In \texttt{libthread\_db.so}, the interface for debugger to export its internal state is undefined
in \texttt{libthread\_db.so}, and this interface should be implemented by the debugger or other utilities,
which would like to use \texttt{libthread\_db.so}.  On \emph{Linux},

\begin{verbatim}
[qiyao@GreenOnly:]$ ldd -r /lib/libthread_db.so.1
undefined symbol: ps_pglobal_lookup     (/lib/libthread_db.so.1)
undefined symbol: ps_pdwrite    (/lib/libthread_db.so.1)
undefined symbol: ps_lsetfpregs (/lib/libthread_db.so.1)
undefined symbol: ps_getpid     (/lib/libthread_db.so.1)
undefined symbol: ps_lsetregs   (/lib/libthread_db.so.1)
undefined symbol: ps_pdread     (/lib/libthread_db.so.1)
undefined symbol: ps_lgetfpregs (/lib/libthread_db.so.1)
undefined symbol: ps_lgetregs   (/lib/libthread_db.so.1)
        linux-gate.so.1 =>  (0x00f9e000)
        libc.so.6 => /lib/libc.so.6 (0x004b8000)
        /lib/ld-linux.so.2 (0x0049b000)
\end{verbatim}

we could find that there are eight functions undefined in \texttt{libthread\_db.so}, and \emph{GDB} defines these
functions in \texttt{gdb/proc-service.c}.  Otherwise, it would fail to link \texttt{libthead\_db.so} at link-time
(via \emph{-lthread\_db} in command line option to linkers) or run-time(via \emph{dlopen}).

When dealing with \texttt{libthread\_db.so}, there are two important data structures, \emph{ps\_prochandle} and 
\emph{td\_thragent\_t}.

Usually, \emph{ps\_prochandle} is also defined at debugger side, instead of \texttt{libthread\_do.so} \footnote{I don't know why.  Need some investigation here}.



Let us look how \emph{GDB} use \emph{ps\_pglobal\_lookup}, here is a snip for debugging \emph{GDB} with \emph{GDB}.
\begin{verbatim}
Breakpoint 3, ps_pglobal_lookup (ph=0x82c1580, obj=0x114324 "libpthread.so.0", name=0x114431 "nptl_version", sym_addr=0xbf908378) at ../../src/gdb/proc-service.c:180
180       ms = lookup_minimal_symbol (name, NULL, NULL);
\end{verbatim}

We could read the \texttt{proc-service.c},
\begin{verbatim}
169 /* Search for the symbol named NAME within the object named OBJ within
170    the target process PH.  If the symbol is found the address of the
171    symbol is stored in SYM_ADDR.  */
172
173 ps_err_e
174 ps_pglobal_lookup (gdb_ps_prochandle_t ph, const char *obj,
175                    const char *name, paddr_t *sym_addr)
176 {
177   struct minimal_symbol *ms;
178
179   /* FIXME: kettenis/2000-09-03: What should we do with OBJ?  */
180   ms = lookup_minimal_symbol (name, NULL, NULL);
181   if (ms == NULL)
182     return PS_NOSYM;
183
184   *sym_addr = SYMBOL_VALUE_ADDRESS (ms);
185   return PS_OK;
186 }
\end{verbatim}

From this piece of code, we could find that, if we want to write a tool to use the functionality
from \texttt{libthread\_db.so}, we should implement the all the stuff about symbol.

It is relatively easy to implement \emph{lookup\_minimal\_symbol}, of course, ignore the efficiency
here.  In ELF specification, we could know that, there is a section named \texttt{.symtab} or \texttt{.dynsym},
and the type of this section is \texttt{SYMTAB} or \texttt{DYNSYM} respecitvely.  User could check these
information via \texttt{readelf -S}

\begin{verbatim}
[qiyao@GreenOnly:~/SourceCode/NGSTAT]$ readelf -S thread-analysis-tool
There are 37 section headers, starting at offset 0x20ff4:

Section Headers:
  [Nr] Name              Type            Addr     Off    Size   ES Flg Lk Inf Al
......
  [ 4] .dynsym           DYNSYM          08048204 000204 0001c0 10   A  5   1  4
.....
  [35] .symtab           SYMTAB          00000000 0215bc 0007d0 10     36  71  4
\end{verbatim}


These information about symbols is well-structured and fixed in length, and they could be easily parsed.

\subsection{Access Target Memory}

The operating system and system library only provide a simple interface to access the state of the debuggee,
and debugger write should wrap this interface, and enrich it.  When dealing with multithreaded program,
debugger writer should be careful.  Since all the libraries in \emph{Linux} are loaded in a lazy mode,
we must make sure that the thread library has been loaded by \emph{loader, ld-2.4.so} when we want to access
the library.

One lesson I learned is that, I want to  get the address of one symbol in thread library.(\emph{nptl\_version}
in \emph{ld-2.4.so}), and try to fetch the content directly.  After nearly one day debugging, I realize that
the \emph{libpthread} has not been loaded by loader.  That is to say, the memory space of \emph{libpthread} could
only be accessed when loader loads it into the process's space.  If the program reference the symbols in \emph{libpthread}
at first, debugger could access the space of \emph{libpthread}

\subsection{Read and Write Registers}
After implement \emph{ps\_get\_thread\_area} in host, the debugger could do some work at the first several steps, with the help of
\emph{ps\_pglobal\_lookup}, \emph{ps\_pdread}, and \emph{ps\_pdwrite}.  Now the debugger could work until operations to read and
write registers are needed.  Since we are dealing with register read or write operations in multithreaed programming enviroment,
so these operations are a little different with that in common conditoins\footnote{I am not sure what is the difference.}.

In \emph{GDB}, \emph{ps\_lgetregs} is written like this,
\begin{verbatim}
/* Get the general registers of LWP LWPID within the target process PH
   and store them in GREGSET.  */

ps_err_e
ps_lgetregs (gdb_ps_prochandle_t ph, lwpid_t lwpid, prgregset_t gregset)
{
  struct cleanup *old_chain = save_inferior_ptid ();

  inferior_ptid = BUILD_LWP (lwpid, ph->pid);

  target_fetch_registers (-1);
  fill_gregset ((gdb_gregset_t *) gregset, -1);

  do_cleanups (old_chain);
  return PS_OK;
}
\end{verbatim}
In \texttt{target.h}, \emph{target\_fetch\_registers} is a macro like this,
\begin{verbatim}
#define	target_fetch_registers(regno)	\
     (*current_target.to_fetch_registers) (regno)
\end{verbatim}

\emph{GDB} will switch \emph{current\_target} to a \textbf{thread specific target} when it begins
to deal with thread, and \texttt{linux-thread-db.c} defines this \textbf{target}.

Set a breakpoint on \emph{thread\_db\_fetch\_registers}, and have a look how it works.
\begin{verbatim}
(top-gdb) bt
#0  ps_lgetregs (ph=0x82c1740, lwpid=7475, gregset=0xbfd07bb0) at ../../src/gdb/proc-service.c:252
#1  0x00111731 in td_ta_map_lwp2thr () from /lib/libthread_db.so.1
#2  0x0011182d in iterate_thread_list () from /lib/libthread_db.so.1
#3  0x00111a85 in td_ta_thr_iter () from /lib/libthread_db.so.1
#4  0x080961d5 in thread_db_find_new_threads () at ../../src/gdb/linux-thread-db.c:1190
\end{verbatim}

In \emph{GDB}, \emph{inferior\_ptid} is a global variable that identify the current thread that monitored
by \emph{GDB}, and it is hard to know which part or component of \emph{GDB}(only about Linux/i386) modify this global variable\footnote{
I try to set watchpoint on \emph{inferior\_ptid} to have a look who modify it, but failed.}.

It seems that I underestimate the complexity of \emph{inferior\_ptid} in \emph{GDB} source tree.  According
to the list of source files that have \textbf{write operation} on it, I find that it is a global variable
that closely coupled with the current structure of \emph{GDB}.

\section{From debugger to libthread\_db}

How debugger make use of \texttt{libthread\_db.so}'s functionality.
\texttt{libthread\_db.so} provides some interfaces for debugger to call to get some internal
events of thread library.  Usually, debugger uses these APIs via \texttt{dlsym}.

\section{attach thread}
In gdb, it seems that "attach a thread" and "attach process" are different.  My understanding is, attach operation in ptrace is
"attach process", but I do not what is "attach a thread", since I am still not clear about the thread model in glibc.



\chapter{Other}
This chapter is about some sections that I don't know where should put them.

\section{inferior call}

\subsection {什么是inferiro call}
在解释\textit{inferior call}的实现原理之前,我们先解释一下\textit{inferior call}是什么。
\textit{inferior call} 可以被简单地理解为“调试器在当前的堆栈上,主动调用被调试程序的函数,并且正确的得到
函数的返回值(这一句好像有点多余)”。\textit{inferior call}理解起来,远比我们想象的简单,这个功能在我们的
调试过程经常会被用到。比如有这样一段代码片段,

\lstinputlisting[language=C,caption={infcall.c},captionpos={b},label={listing:other:infcall},frame=trbl]{other/infcall.c}

假设现在程序执行到\texttt{foo}中,即将调用\texttt{bar},我们想检查给定一个输入,函数的\texttt{bar}的返回值是否正确。通常这个
时候我们想在GDB这里调用一下目标代码中的函数,
\begin{verbatim}
(gdb) p bar (5)
$1 = 4
\end{verbatim} 

其实\texttt{bar (5)}是在目标端运行的,这就是一个\textit{inferior call}。

在有些程序中,为了方便调试,开发人员会开发一些调试函数,留在程序中。这样在调试阶段,就可以利用\textit{inferior call}这个功能,把一些程序的内部状态给打印出来。比如在调试GCC rtl的时候,我们经常在GDB中用 \texttt{p print\_rtx (rtx)},来打印\texttt{rtx}
的内容。这就是一个\textit{inferior call}。

\subsection {inferiro call的原理}
在知道了\textit{inferior call}到底是什么的时候,我们来仔细想一下,第一,函数\texttt{print\_rtx (rtx)}是在被调试程序那里执行的,
而不是在GDB这里;第二,参数\textit{rtx}是GDB传给被调试程序的函数;第三,GDB能够正确得到\texttt{print\_rtx (rtx)}这个函数
的返回值;最后,被调试程序在执行完成\texttt{print\_rtx (rtx)}后,仍然停在原来的位置,程序的状态没有任何变化。上边这几点,就是
调试器为了实现\textit{inferior call}所需要做到的,或者,需要保证的。

下来我们想想,调试器应该怎样保证上边四条,
\begin{enumerate}
  \item 第一,函数是在被调试程序那里实行的。这个容易实现,调试器需要找到函数的入口地址,把寄存器\texttt{pc}设置为函数的入口地址,
    然后用\texttt{ptrace}让被调试程序继续执行(\texttt{PTRACE\_CONT})就可以了。
  \item 第二,调试器传递参数给调用的函数。这个相对麻烦一些,因为调试器需要知道当前被调试程序所在系统的函数调用规则(function call
convention),也就是规定,每一个函数的参数,应该如何传递给被调用函数。这个也是系统ABI的一部分。简单点说,如果函数有四个参数,这个四
个参数在函数调用的时候,应该在什么位置,是在寄存器里边还是在堆栈上?调试器在知道这些规范以后,必须按照规范把被调用函数需要的参数,
放在规定的地方。
  \item 第三, 调试器需要得到函数调用的返回值。函数调用的返回值所在的位置同样规范在系统的ABI中。函数的返回值是在寄存器里边,还是堆栈上?
调试器根据这个规范,从规定的位置,得到函数的返回值。
  \item 第四,调试器需要函数调用完成后,程序依然停在之前的位置,没有任何变化。被调用的函数是不知道它被调用是一般的正常调用,还是调试器
用\textit{inferior call}。所以,调试器在程序返回地址上,做了一些手脚。一般来说,程序的返回地址应该是调用这个函数的指令的下一条指令。
在\textit{inferior call},调试器会把函数调用的返回地址设置为一个特殊的地址,并且在这个地址上设置一个断点。这样,当程序返回的时候,调试器
就能够保证程序执行完成后,程序就停下来。
\end{enumerate}

这样看来,只要能够对系统的ABI有个全面的了解,实现\textit{inferior call}应该没有什么困难。

\subsection {inferiro call在GDB中的实现}
下来我们看看GDB里边,是怎么实现
\textit{inferior call}的,
\begin{table}
  \begin{tabular}{|l|l|l|} \hline
    GDB hook&输入&作用\\ \hline
    \tiny\textbf{arch}\texttt{\_push\_dummy\_call}&\tiny type parameter and return.&\tiny 根据ABI把参数放到应该放的地方\\&\tiny param value&\\ \hline

    \tiny\textbf{arch}\texttt{\_return\_value}&\tiny 返回值类型&\tiny 根据ABI从正确位置取得返回值\\ \hline

    \tiny\textbf{arch}\texttt{\_frame\_align}&&\tiny frame alignment\\ \hline

    \tiny\textbf{arch}\texttt{\_value\_to\_register}&& \tiny 寄存器和\texttt{Value}之间的转换\\
    \tiny\textbf{arch}\texttt{\_register\_to\_value}&&\\ \hline
    \tiny\textbf{arch}\texttt{\_push\_dummy\_id} &&\tiny 建立一个dummy \texttt{frame id}\\ \hline
  \end{tabular}
\end{table}

\textit{inferior call}在GDB中的实现,相对模块化,没有与其他的部分有过多的联系,但是我们可以看到在GDB的实现中,引入了一个\textit{dummy frame}
这个概念。这个概念和GDB的frame模块有关系,和\textit{inferior call}功能没有实质联系。\textit{dummy frame}只是为了考虑当目标程序进行inferior call的时候,被中断(比如遇到断点或者信号),程序的frame看着依然正常。如果把\textit{dummy frame}从\textit{inferior call}部分中去掉,没有任何实际影响,只是在有些情况下,程序的stack backtrace 可能有些奇怪或者难以理解。

\section{signal trampoline frame}

%%%%%%%%%%%%%%%%%%%%%%%%%%%%%%%%%%%%%%%%%%%%%%%%%%%%%%%%%%%%%%%%%%%%%%%%%%%%%%%%%%%
\renewcommand{\bibname}{Reference}
\bibliography{reference}
\bibliographystyle{plain}
\addcontentsline{toc}{chapter}{Reference}

\end{CJK}
\end{document} 
