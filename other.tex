\chapter{Other}
This chapter is about some sections that I don't know where should put them.

\section{inferior call}

\subsection {什么是inferiro call}
在解释\textit{inferior call}的实现原理之前,我们先解释一下\textit{inferior call}是什么。
\textit{inferior call} 可以被简单地理解为“调试器在当前的堆栈上,主动调用被调试程序的函数,并且正确的得到
函数的返回值(这一句好像有点多余)”。\textit{inferior call}理解起来,远比我们想象的简单,这个功能在我们的
调试过程经常会被用到。比如有这样一段代码片段,

\lstinputlisting[language=C,caption={infcall.c},captionpos={b},label={listing:other:infcall},frame=trbl]{other/infcall.c}

假设现在程序执行到\texttt{foo}中,即将调用\texttt{bar},我们想检查给定一个输入,函数的\texttt{bar}的返回值是否正确。通常这个
时候我们想在GDB这里调用一下目标代码中的函数,
\begin{verbatim}
(gdb) p bar (5)
$1 = 4
\end{verbatim} 

其实\texttt{bar (5)}是在目标端运行的,这就是一个\textit{inferior call}。

在有些程序中,为了方便调试,开发人员会开发一些调试函数,留在程序中。这样在调试阶段,就可以利用\textit{inferior call}这个功能,把一些程序的内部状态给打印出来。比如在调试GCC rtl的时候,我们经常在GDB中用 \texttt{p print\_rtx (rtx)},来打印\texttt{rtx}
的内容。这就是一个\textit{inferior call}。

\subsection {inferiro call的原理}
在知道了\textit{inferior call}到底是什么的时候,我们来仔细想一下,第一,函数\texttt{print\_rtx (rtx)}是在被调试程序那里执行的,
而不是在GDB这里;第二,参数\textit{rtx}是GDB传给被调试程序的函数;第三,GDB能够正确得到\texttt{print\_rtx (rtx)}这个函数
的返回值;最后,被调试程序在执行完成\texttt{print\_rtx (rtx)}后,仍然停在原来的位置,程序的状态没有任何变化。上边这几点,就是
调试器为了实现\textit{inferior call}所需要做到的,或者,需要保证的。

下来我们想想,调试器应该怎样保证上边四条,
\begin{enumerate}
  \item 第一,函数是在被调试程序那里实行的。这个容易实现,调试器需要找到函数的入口地址,把寄存器\texttt{pc}设置为函数的入口地址,
    然后用\texttt{ptrace}让被调试程序继续执行(\texttt{PTRACE\_CONT})就可以了。
  \item 第二,调试器传递参数给调用的函数。这个相对麻烦一些,因为调试器需要知道当前被调试程序所在系统的函数调用规则(function call
convention),也就是规定,每一个函数的参数,应该如何传递给被调用函数。这个也是系统ABI的一部分。简单点说,如果函数有四个参数,这个四
个参数在函数调用的时候,应该在什么位置,是在寄存器里边还是在堆栈上?调试器在知道这些规范以后,必须按照规范把被调用函数需要的参数,
放在规定的地方。
  \item 第三, 调试器需要得到函数调用的返回值。函数调用的返回值所在的位置同样规范在系统的ABI中。函数的返回值是在寄存器里边,还是堆栈上?
调试器根据这个规范,从规定的位置,得到函数的返回值。
  \item 第四,调试器需要函数调用完成后,程序依然停在之前的位置,没有任何变化。被调用的函数是不知道它被调用是一般的正常调用,还是调试器
用\textit{inferior call}。所以,调试器在程序返回地址上,做了一些手脚。一般来说,程序的返回地址应该是调用这个函数的指令的下一条指令。
在\textit{inferior call},调试器会把函数调用的返回地址设置为一个特殊的地址,并且在这个地址上设置一个断点。这样,当程序返回的时候,调试器
就能够保证程序执行完成后,程序就停下来。
\end{enumerate}

这样看来,只要能够对系统的ABI有个全面的了解,实现\textit{inferior call}应该没有什么困难。

\subsection {inferiro call在GDB中的实现}
下来我们看看GDB里边,是怎么实现
\textit{inferior call}的,
\begin{table}
  \begin{tabular}{|l|l|l|} \hline
    GDB hook&输入&作用\\ \hline
    \tiny\textbf{arch}\texttt{\_push\_dummy\_call}&\tiny type parameter and return.&\tiny 根据ABI把参数放到应该放的地方\\&\tiny param value&\\ \hline

    \tiny\textbf{arch}\texttt{\_return\_value}&\tiny 返回值类型&\tiny 根据ABI从正确位置取得返回值\\ \hline

    \tiny\textbf{arch}\texttt{\_frame\_align}&&\tiny frame alignment\\ \hline

    \tiny\textbf{arch}\texttt{\_value\_to\_register}&& \tiny 寄存器和\texttt{Value}之间的转换\\
    \tiny\textbf{arch}\texttt{\_register\_to\_value}&&\\ \hline
    \tiny\textbf{arch}\texttt{\_push\_dummy\_id} &&\tiny 建立一个dummy \texttt{frame id}\\ \hline
  \end{tabular}
\end{table}

\textit{inferior call}在GDB中的实现,相对模块化,没有与其他的部分有过多的联系,但是我们可以看到在GDB的实现中,引入了一个\textit{dummy frame}
这个概念。这个概念和GDB的frame模块有关系,和\textit{inferior call}功能没有实质联系。\textit{dummy frame}只是为了考虑当目标程序进行inferior call的时候,被中断(比如遇到断点或者信号),程序的frame看着依然正常。如果把\textit{dummy frame}从\textit{inferior call}部分中去掉,没有任何实际影响,只是在有些情况下,程序的stack backtrace 可能有些奇怪或者难以理解。

\section{signal trampoline frame}